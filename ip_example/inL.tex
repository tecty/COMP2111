\documentclass[a4paper,12pt,fleqn]{scrartcl}
  \usepackage[l2tabu,orthodox]{nag}% Old habits die hard. All the same, there are commands, classes and packages which are outdated and superseded. nag provides routines to warn the user about the use of those.
  
  \usepackage[all,error]{onlyamsmath}% Error on deprecated math commands like $$ $$.
  \usepackage[strict=true]{csquotes}
  
  %\usepackage{color}
  
  \usepackage{listings}
  \lstset{frame=single,framerule=0pt,language={C},showstringspaces=false,numbers=left,columns=fullflexible}
  
  % COMP2111-specific macros. See
  % http://www.cse.unsw.edu.au/~cs2111/18s1/LaTeX/primer.html
  \usepackage{2111defs2}
  \usepackage[colorlinks=true]{hyperref}
  
  \newcommand{\assn}[1]{{\color{red}\left\{#1\right\}}}
  \newcommand{\remark}[1]{{\sffamily\color{blue}{#1}}}
  
  % define some convenience macros specific to this task
  \newcommand{\perm}{\mathsf{perm}}
  
  \title{Assignment 1}
  % name OR student number for each of the two
  \author{z1234567\and Jack Shit}
  
  \begin{document}
  \maketitle
  
  \remark{I've sprinkled a few explanatory remarks in here, typeset like
    this one. You wouldn't do that to Liam. Instead you'd fill in all
    the gaps.}
  
  \section{Task 1}
  \label{sec:task-1}
  Having defined that array $a$ of length $n$ contains a permutation (as
  described in the assignment 1 (16s1) specification) by
  \begin{gather*}
    \perm(a,n) = \All{k\in 0..(n-1)}{\Exi{\ell\in 0..(n-1)}{a[\ell]=k}}
  \end{gather*}
  we define our precondition
  \begin{gather*}
    \perm(a,n)\And a=a_0
  \end{gather*}
  so that it states that $a$ contains a permutation. We also freeze
  $a$'s initial value for later reference in the postcondition
  \begin{gather*}
    \All{k\in0..(n-1)}{b[a_0[k]]=k}
  \end{gather*}
  which states that $a$ is the inverse of its original value upon
  termination of our program.
  
  %\remark{Could say more here about why this captures the requirements.}
  \section{Task 2}
  \label{sec:task-2}
  
  We propose the following proof outline to demonstrate the correctness
  of our code (in black).
  \begin{gather*}
     \assn{\perm(a,n)\And a=a_0}\\
     \assn{I\subst 0i}\\
     i\Ass 0;\\
     \assn{I}\\
     \WHILE~i<n~\DO\\
     \qquad \assn{I\And i<n}\\
     \qquad \assn{I\subst{i+1}i\subst{(b:a[i]\mapsto i)}b}\\
     \qquad b[a[i]]\Ass i;\\
     \qquad \assn{I\subst{i+1}i}\\
     \qquad i\Ass i+1\\
     \qquad \assn{I}\\
     \OD;\\
     \assn{I\And i\geq n}\\
     \assn{J\subst 0i}\\
     i\Ass 0;\\
     \assn{J}\\
     \WHILE~i<n~\DO\\
     \qquad \assn{J\And i<n}\\
     \qquad \assn{J\subst{i+1}i\subst{(a:i\mapsto b[i])}a}\\
     \qquad a[i]\Ass b[i];\\
     \qquad \assn{J\subst{i+1}i}\\
     \qquad i\Ass i+1\\
     \qquad \assn{J}\\
     \OD;\\
     \assn{J\And i\geq n}\\
     \assn{\All{k\in 0..(n-1)}{a[a_0[k]]=k}}
  \end{gather*}
  where our invariants are
  \begin{align*}
    I &= \perm(a_0,n)\And 0\leq i \leq n \And a = a_0\And \All{k\in 0..(i-1)}{b[a_0[k]]=k}\\
    J &= I\subst{a_0,n}{a,i}\And 0\leq i \leq n\And \All{k\in 0..(i-1)}{a[k]=b[k]}
  \end{align*}
  \remark{Apart from the pre- and postcondition, all assertion as they occur in
  the code were derived using the Hoare logic rules for the enclosed
  constructs. Finding the two invariants is the only marginally creative
  activity here.} The only remaining proof obligations are the five
  implications between adjacent assertions.
  
  \subsection{First implication: $\perm(a,n)\And a=a_0\Implies I\subst 0i$}
  
  \remark{We \emph{find} this proof by beginning with the right-hand-side
    (RHS) of the implication, unfolding the definition of $I$, and performing the
    substitution. We \emph{present} it however in the conventional
    order, namely, from left to right.}
  \begin{align*}
    &\perm(a,n)\And a=a_0\\
    %
    \justification[\Implies]{using $n\in\nat$ and realising that the last conjunct is vacuously true}
    % 
    &\perm(a_0,n)\And 0\leq 0 \leq n \And a = a_0\And \All{k\in 0..(0-1)}{b[a_0[k]]=k}\\
    %
    \justification{definitions of $I$ and substitution}
    %
    &I\subst 0i
  \end{align*}
  
  \subsection{Second implication: $I\And i<n \Implies I\subst{i+1}i\subst{(b:a_0[i]\mapsto i)}b$}
  
  First we expand $I$ and perform the substitutions to arrive at
  \begin{align*}
    &\perm(a_0,n)\And 0\leq i \leq n \And a = a_0\And
      \All{k\in 0..(i-1)}{b[a_0[k]]=k}\And i<n\\
    {}\Implies{}&
      \perm(a_0,n)\And 0\leq (i+1) \leq n \And a = a_0\And {}\\
    & \All{k\in 0..((i+1)-1)}{(b:a_0[i]\mapsto i)[a_0[k]]=k}
  \end{align*}
  We discharge the conjuncts on the RHS separately. The first
  ($\perm(a_0,n)$) and third ($a=a_0$) are both present as conjuncts on
  the LHS. The second ($0\leq (i+1) \leq n$) follows from
  $0\leq i \leq n$ and $i<n$ on the LHS because (by an assumption left
  implicit) $i$ and $n$ are natural numbers. The fourth warrants a more
  careful proof. We simplify it as follows.
  \begin{align*}
    &\All{k\in 0..((i+1)-1)}{(b:a_0[i]\mapsto i)[a_0[k]]=k}\\
    %
    \justification{extracting $i$ from the set over which $k$ ranges %in the quantification
    to deal with it separately}
    %
    & \All{k\in 0..(i-1)}{(b:a_0[i]\mapsto i)[a_0[k]]=k}\And (b:a_0[i]\mapsto i)[a_0[i]]=i\\
    \intertext{The first of these two conjuncts can be simplified by observing that $(b:a_0[i]\mapsto i)[\ell] = b[\ell]$ unless $\ell=a_0[i]$ and that this is never the case for any of the $a_0[k]$ with $0\leq k < i$ because $a_0$ is a permutation. The second can be simplified using $(f:x\mapsto y)(x) = y$.}
    &\Equiv \All{k\in 0..(i-1)}{b[a_0[k]]=k}\And i=i
  \end{align*}
  The first conjunct is present on the LHS while the second is trivially
  true.
  
  \subsection{Third implication: $I\And i\geq n\Implies J\subst 0i$}
  
  \begin{align*}
    & I\And i\geq n\\
    \justification{Def.~of $I$}
    & \perm(a_0,n)\And 0\leq i \leq n \And a = a_0\And \All{k\in 0..(i-1)}{b[a_0[k]]=k}\And i\geq n\\
    \intertext{\remark{stuff missing here}}
    % ...
    & I\subst{a_0,n}{a,i}\And 0\leq 0 \leq n\And \All{k\in 0..(0-1)}{a[k]=b[k]}\\
    \justification{Def's of $J$ and substitution} 
    & J\subst 0i
  \end{align*}
  
  
  \subsection{Fifth implication: $J\And i\geq n\Implies\All{k\in 0..(n-1)}{a[a_0[k]]=k}$}
  \label{sec:fifth-impl-assnj}
  
  \begin{align*}
    & J\And i\geq n\\
    \justification{Def.~of $J$}
    & I\subst{a_0,n}{a,i}\And 0\leq i \leq n\And \All{k\in 0..(i-1)}{a[k]=b[k]}\And i\geq n\\
    \justification{Def.~of $I$ and substitution}
    & \perm(a_0,n)\And 0\leq n \leq n \And a_0 = a_0\And \All{k\in 0..(n-1)}{b[a_0[k]]=k}\And{}\\
    & 0\leq i \leq n\And \All{k\in 0..(i-1)}{a[k]=b[k]}\And i\geq n\\
  %
    \justification[\Implies]{simplify}
    & \perm(a_0,n)\And 0\leq n \leq n \And a_0 = a_0\And \All{k\in 0..(n-1)}{b[a_0[k]]=k}\And{}\\
    & \All{k\in 0..(n-1)}{a[k]=b[k]}\\
    %
    \justification[\Implies]{by the second line, ``$a = b$'', we use the last conjunct of the first to obtain}
    & \All{k\in 0..(n-1)}{a[a_0[k]]=k}  
  \end{align*}
  
  \section{Task 3}
  \label{sec:task-3}
  
  % for assignment 1 (17s1), you don't implement barf - it's provided
  \lstinputlisting{ip.c}
  
  \remark{Should discuss here the differences between the toy language
    code and the C code, including:}
  
  In our C implementation, we opted for the more traditional
  \lstinline{for} loop idiom instead of a \lstinline{while} loop. It
  should be clear that our \lstinline{for} loop captures the meaning of
  the entire first half of the toy language program. As for the second
  half, we used a call of the C library function \lstinline{memcpy} to
  implement the second loop (pseudo-code ``$a\Ass b$'') that copies the
  content of array $b$ into array $a$. (Cf.~\lstinline{man memcpy}.)
  \end{document}